\documentclass[12pt]{article}
\usepackage[margin=1.0in]{geometry}
\usepackage{amssymb}

%title material
\title{ASTR 400B Homework 3 Solutions}
\date{}

%begin the document
\begin{document}

%make the title, goes after document begins
\maketitle


\begin{table}[h]
\centering
\caption{ {\bf Mass Break Down of the Local Group} }
\label{table:Masses}
\begin{tabular}{ |c|c|c|c|c|c| }
\hline\hline
Galaxy Name &  Halo Mass & Disk Mass & Bulge Mass & Total & $f_{\rm bar}$\\   
	    &  ($10^{12}$ M$_\odot$)   & ($10^{12}$ M$_\odot$)  & ($10^{12}$ M$_\odot$) & ($10^{12}$ M$_\odot$) &  \\
\hline
MW  &  1.975  &  0.075 & 0.010 & 2.060  & 4\% \\
M31 &  1.921 & 0.120 & 0.019 & 2.060 & 7\% \\
M33 &  0.187  &  0.009  &  N/A & 0.196 &  5\% \\
\hline
Local Group &  &  &  &  4.316  &   5\% \\
\hline\hline
\end{tabular}
\end{table}

\begin{enumerate}

\item  The total mass of the MW and M31 is equivalent in this model.  The halo mass dominates the mass budget. 

\item  M$_{M31\ast}$ / M$_{MW\ast} \sim$ 1.6. The stellar mass of M31 is larger than the MW; M31 is more luminous.

\item  The dark matter halo masses of the MW and M31 are $\sim$ equivalent.  We naively expect there to be a relationship between the dark matter mass of a galaxy and its stellar mass, where more luminous galaxies reside in more massive halos. This is called ``abundance matching"  - but that relationship is not linear and also has a lot of scatter. This simulation used one set of values for the dark matter halo mass that is allowed by the scatter.  Other solutions exist, but we don't expect the dark matter mass of the MW and M31 to differ by more than a factor of 2.

\item The baryon fraction of each galaxy is much less than the cosmologically expected baryon fraction. Most spiral galaxies have baryon fractions of 4-7\%. 
Smaller (dwarf) galaxies have even smaller baryon fractions. 
  Reasons: 1) Outflows may feed the
Circumgalactic medium (CGM), removing baryons from the disk;  2) Gas may be inefficiently accreted onto galactic disks, staying in the CGM instead of 
contributing to the disk.  The total mass in the CGM of the MW is poorly constrained because it exists at multiple temperatures, from 10$^6$ K (hot; x-ray gas), to 
the Warm/Hot Intergalactic Medium (WHIM) at 10$^5$ K and cooler gas at 10$^4 $K (both ionized gas and neutral HI).  We do not yet have a full census of the 
amount of gas in the CGM, but current estimates suggest we can make up the ``missing baryons" in the CGM.  
 


\end{enumerate}




\end{document}